%Version 2.1 April 2023
% See section 11 of the User Manual for version history
%
%%%%%%%%%%%%%%%%%%%%%%%%%%%%%%%%%%%%%%%%%%%%%%%%%%%%%%%%%%%%%%%%%%%%%%
%%                                                                 %%
%% Please do not use \input{...} to include other tex files.       %%
%% Submit your LaTeX manuscript as one .tex document.              %%
%%                                                                 %%
%% All additional figures and files should be attached             %%
%% separately and not embedded in the \TeX\ document itself.       %%
%%                                                                 %%
%%%%%%%%%%%%%%%%%%%%%%%%%%%%%%%%%%%%%%%%%%%%%%%%%%%%%%%%%%%%%%%%%%%%%

\documentclass[sn-basic, lineno,pdflatex]{sn-jnl}

%%%% Standard Packages
%%<additional latex packages if required can be included here>

\usepackage{graphicx}%
\usepackage{multirow}%
\usepackage{amsmath,amssymb,amsfonts}%
\usepackage{amsthm}%
\usepackage{mathrsfs}%
\usepackage[title]{appendix}%
\usepackage{xcolor}%
\usepackage{textcomp}%
\usepackage{manyfoot}%
\usepackage{booktabs}%
\usepackage{algorithm}%
\usepackage{algorithmicx}%
\usepackage{algpseudocode}%
\usepackage{listings}%
%%%%

%%%%%=============================================================================%%%%
%%%%  Remarks: This template is provided to aid authors with the preparation
%%%%  of original research articles intended for submission to journals published
%%%%  by Springer Nature. The guidance has been prepared in partnership with
%%%%  production teams to conform to Springer Nature technical requirements.
%%%%  Editorial and presentation requirements differ among journal portfolios and
%%%%  research disciplines. You may find sections in this template are irrelevant
%%%%  to your work and are empowered to omit any such section if allowed by the
%%%%  journal you intend to submit to. The submission guidelines and policies
%%%%  of the journal take precedence. A detailed User Manual is available in the
%%%%  template package for technical guidance.
%%%%%=============================================================================%%%%



\raggedbottom



% Pandoc syntax highlighting
\usepackage{color}
\usepackage{fancyvrb}
\newcommand{\VerbBar}{|}
\newcommand{\VERB}{\Verb[commandchars=\\\{\}]}
\DefineVerbatimEnvironment{Highlighting}{Verbatim}{commandchars=\\\{\}}
% Add ',fontsize=\small' for more characters per line
\usepackage{framed}
\definecolor{shadecolor}{RGB}{248,248,248}
\newenvironment{Shaded}{\begin{snugshade}}{\end{snugshade}}
\newcommand{\AlertTok}[1]{\textcolor[rgb]{0.94,0.16,0.16}{#1}}
\newcommand{\AnnotationTok}[1]{\textcolor[rgb]{0.56,0.35,0.01}{\textbf{\textit{#1}}}}
\newcommand{\AttributeTok}[1]{\textcolor[rgb]{0.13,0.29,0.53}{#1}}
\newcommand{\BaseNTok}[1]{\textcolor[rgb]{0.00,0.00,0.81}{#1}}
\newcommand{\BuiltInTok}[1]{#1}
\newcommand{\CharTok}[1]{\textcolor[rgb]{0.31,0.60,0.02}{#1}}
\newcommand{\CommentTok}[1]{\textcolor[rgb]{0.56,0.35,0.01}{\textit{#1}}}
\newcommand{\CommentVarTok}[1]{\textcolor[rgb]{0.56,0.35,0.01}{\textbf{\textit{#1}}}}
\newcommand{\ConstantTok}[1]{\textcolor[rgb]{0.56,0.35,0.01}{#1}}
\newcommand{\ControlFlowTok}[1]{\textcolor[rgb]{0.13,0.29,0.53}{\textbf{#1}}}
\newcommand{\DataTypeTok}[1]{\textcolor[rgb]{0.13,0.29,0.53}{#1}}
\newcommand{\DecValTok}[1]{\textcolor[rgb]{0.00,0.00,0.81}{#1}}
\newcommand{\DocumentationTok}[1]{\textcolor[rgb]{0.56,0.35,0.01}{\textbf{\textit{#1}}}}
\newcommand{\ErrorTok}[1]{\textcolor[rgb]{0.64,0.00,0.00}{\textbf{#1}}}
\newcommand{\ExtensionTok}[1]{#1}
\newcommand{\FloatTok}[1]{\textcolor[rgb]{0.00,0.00,0.81}{#1}}
\newcommand{\FunctionTok}[1]{\textcolor[rgb]{0.13,0.29,0.53}{\textbf{#1}}}
\newcommand{\ImportTok}[1]{#1}
\newcommand{\InformationTok}[1]{\textcolor[rgb]{0.56,0.35,0.01}{\textbf{\textit{#1}}}}
\newcommand{\KeywordTok}[1]{\textcolor[rgb]{0.13,0.29,0.53}{\textbf{#1}}}
\newcommand{\NormalTok}[1]{#1}
\newcommand{\OperatorTok}[1]{\textcolor[rgb]{0.81,0.36,0.00}{\textbf{#1}}}
\newcommand{\OtherTok}[1]{\textcolor[rgb]{0.56,0.35,0.01}{#1}}
\newcommand{\PreprocessorTok}[1]{\textcolor[rgb]{0.56,0.35,0.01}{\textit{#1}}}
\newcommand{\RegionMarkerTok}[1]{#1}
\newcommand{\SpecialCharTok}[1]{\textcolor[rgb]{0.81,0.36,0.00}{\textbf{#1}}}
\newcommand{\SpecialStringTok}[1]{\textcolor[rgb]{0.31,0.60,0.02}{#1}}
\newcommand{\StringTok}[1]{\textcolor[rgb]{0.31,0.60,0.02}{#1}}
\newcommand{\VariableTok}[1]{\textcolor[rgb]{0.00,0.00,0.00}{#1}}
\newcommand{\VerbatimStringTok}[1]{\textcolor[rgb]{0.31,0.60,0.02}{#1}}
\newcommand{\WarningTok}[1]{\textcolor[rgb]{0.56,0.35,0.01}{\textbf{\textit{#1}}}}

% tightlist command for lists without linebreak
\providecommand{\tightlist}{%
  \setlength{\itemsep}{0pt}\setlength{\parskip}{0pt}}

% From pandoc table feature
\usepackage{longtable,booktabs,array}
\usepackage{calc} % for calculating minipage widths
% Correct order of tables after \paragraph or \subparagraph
\usepackage{etoolbox}
\makeatletter
\patchcmd\longtable{\par}{\if@noskipsec\mbox{}\fi\par}{}{}
\makeatother
% Allow footnotes in longtable head/foot
\IfFileExists{footnotehyper.sty}{\usepackage{footnotehyper}}{\usepackage{footnote}}
\makesavenoteenv{longtable}




\begin{document}


\title[Check your outliers]{Check your outliers! An introduction to
identifying statistical outliers in R with \emph{easystats}}

%%=============================================================%%
%% Prefix	-> \pfx{Dr}
%% GivenName	-> \fnm{Joergen W.}
%% Particle	-> \spfx{van der} -> surname prefix
%% FamilyName	-> \sur{Ploeg}
%% Suffix	-> \sfx{IV}
%% NatureName	-> \tanm{Poet Laureate} -> Title after name
%% Degrees	-> \dgr{MSc, PhD}
%% \author*[1,2]{\pfx{Dr} \fnm{Joergen W.} \spfx{van der} \sur{Ploeg} \sfx{IV} \tanm{Poet Laureate}
%%                 \dgr{MSc, PhD}}\email{iauthor@gmail.com}
%%=============================================================%%

\author*[1]{\fnm{Rémi} \sur{Thériault} }\email{\href{mailto:theriault.remi@courrier.uqam.ca}{\nolinkurl{theriault.remi@courrier.uqam.ca}}}

\author[2]{\fnm{Mattan} \spfx{S.} \sur{Ben-Shachar} }

\author[3]{\fnm{Indrajeet} \sur{Patil} }

\author[4]{\fnm{Daniel} \sur{Lüdecke} }

\author[5]{\fnm{Brenton} \spfx{M.} \sur{Wiernik} }

\author[6]{\fnm{Dominique} \sur{Makowski} }



  \affil[1]{\orgname{Department of Psychology, Université du Québec à
Montréal, Montréal, Québec, Canada}}
  \affil[2]{\orgname{Independent Researcher, Ramat Gan, Israel}}
  \affil[3]{\orgname{Center for Humans and Machines, Max Planck
Institute for Human Development, Berlin, Germany}}
  \affil[4]{\orgname{Institute of Medical Sociology, University Medical
Center Hamburg-Eppendorf, Germany}}
  \affil[5]{\orgname{Independent Researcher, Tampa, FL, USA}}
  \affil[6]{\orgname{School of Psychology, University of Sussex,
Brighton, UK}}

\abstract{Beyond the challenge of keeping up-to-date with current best
practices regarding the diagnosis and treatment of outliers, an
additional difficulty arises concerning the mathematical implementation
of the recommended methods. Here, we provide an overview of current
recommendations and best practices and demonstrate how they can easily
and conveniently be implemented in the R statistical computing software,
using the \emph{\{performance\}} package of the \emph{easystats}
ecosystem. We cover univariate, multivariate, and model-based
statistical outlier detection methods, their recommended threshold,
standard output, and plotting methods. We conclude by reviewing the
different theoretical types of outliers, whether to exclude or winsorize
them, and the importance of transparency.}

\keywords{univariate outliers; multivariate outliers; robust detection
methods; R; easystats}



\maketitle

\hypertarget{introduction}{%
\section{Introduction}\label{introduction}}

Real-life data often contain observations that can be considered
\emph{abnormal} when compared to the main population. The cause of it
can be hard to assess and the boundaries of ``abnormal'', difficult to
define---they may belong to a different distribution (originating from a
different generative process) or simply be extreme cases, statistically
rare but not impossible.

Nonetheless, the improper handling of these outliers can substantially
affect statistical model estimations, biasing effect estimations and
weakening the models' predictive performance. It is thus essential to
address this problem in a thoughtful manner. Yet, despite the existence
of established recommendations and guidelines, many researchers still do
not treat outliers in a consistent manner, or do so using inappropriate
strategies \citep{simmons2011false, leys2013outliers}.

One possible reason is that researchers are not aware of the existing
recommendations, or do not know how to implement them using their
analysis software. In this paper, we show how to follow current best
practices for automatic and reproducible statistical outlier detection
(SOD) using R and the \emph{\{performance\}} package
\citep{ludecke2021performance}, which is part of the \emph{easystats}
ecosystem of packages that build an R framework for easy statistical
modeling, visualization, and reporting \citep{easystatspackage}.
Installation instructions can be found on
\href{https://github.com/easystats/performance}{GitHub} or its
\href{https://easystats.github.io/performance/}{website}, and its list
of dependencies on
\href{https://cran.r-project.org/package=performance}{CRAN}.

The instructional materials that follow are aimed at an audience of
researchers who want to follow good practices, and are appropriate for
advanced undergraduate students, graduate students, professors, or
professionals having to deal with the nuances of outlier treatment.

\hypertarget{identifying-outliers}{%
\section{Identifying Outliers}\label{identifying-outliers}}

Although many researchers attempt to identify outliers with measures
based on the mean (e.g., \emph{z} scores), those methods are problematic
because the mean and standard deviation themselves are not robust to the
influence of outliers and those methods also assume normally distributed
data (i.e., a Gaussian distribution). Therefore, current guidelines
recommend using robust methods to identify outliers, such as those
relying on the median as opposed to the mean
\citep{leys2019outliers, leys2013outliers, leys2018outliers}.

Nonetheless, which exact outlier method to use depends on many factors.
In some cases, eye-gauging odd observations can be an appropriate
solution, though many researchers will favour algorithmic solutions to
detect potential outliers, for example, based on a continuous value
expressing the observation stands out from the others.

One of the factors to consider when selecting an algorithmic outlier
detection method is the statistical test of interest. Identifying
observations the regression model does not fit well can help find
information relevant to our specific research context. This approach,
known as model-based outliers detection (as outliers are extracted after
the statistical model has been fit), can be contrasted with
distribution-based outliers detection, which is based on the distance
between an observation and the ``center'' of its population. Various
quantification strategies of this distance exist for the latter, both
univariate (involving only one variable at a time) or multivariate
(involving multiple variables).

When no method is readily available to detect model-based outliers, such
as for structural equation modelling (SEM), looking for multivariate
outliers may be of relevance. For simple tests (\emph{t} tests or
correlations) that compare values of the same variable, it can be
appropriate to check for univariate outliers. However, univariate
methods can give false positives since \emph{t} tests and correlations,
ultimately, are also models/multivariable statistics. They are in this
sense more limited, but we show them nonetheless for educational
purposes.

Importantly, whatever approach researchers choose remains a subjective
decision, which usage (and rationale) must be transparently documented
and reproducible \citep{leys2019outliers}. Researchers should commit
(ideally in a preregistration) to an outlier treatment method before
collecting the data. They should report in the paper their decisions and
details of their methods, as well as any deviation from their original
plan. These transparency practices can help reduce false positives due
to excessive researchers' degrees of freedom (i.e., choice flexibility
throughout the analysis). In the following section, we will go through
each of the mentioned methods and provide examples on how to implement
them with R.

\hypertarget{univariate-outliers}{%
\subsection{Univariate Outliers}\label{univariate-outliers}}

Researchers frequently attempt to identify outliers using measures of
deviation from the center of a variable's distribution. One of the most
popular such procedure is the \emph{z} score transformation, which
computes the distance in standard deviation (SD) from the mean. However,
as mentioned earlier, this popular method is not robust. Therefore, for
univariate outliers, it is recommended to use the median along with the
Median Absolute Deviation (MAD), which are more robust than the
interquartile range or the mean and its standard deviation
\citep{leys2019outliers, leys2013outliers}.

Researchers can identify outliers based on robust (i.e., MAD-based)
\emph{z} scores using the \texttt{check\_outliers()} function of the
\emph{\{performance\}} package, by specifying
\texttt{method\ =\ "zscore\_robust"}.\footnote{Note that
  \texttt{check\_outliers()} only checks numeric variables.} Although
\citet{leys2013outliers} suggest a default threshold of 2.5 and
\citet{leys2019outliers} a threshold of 3, \emph{\{performance\}} uses
by default a less conservative threshold of
\textasciitilde3.29.\footnote{3.29 is an approximation of the two-tailed
  critical value for \emph{p} \textless{} .001, obtained through
  \texttt{qnorm(p\ =\ 1\ -\ 0.001\ /\ 2)}. We chose this threshold for
  consistency with the thresholds of all our other methods.} That is,
data points will be flagged as outliers if they go beyond +/-
\textasciitilde3.29 MAD. Users can adjust this threshold using the
\texttt{threshold} argument.

Below we provide example code using the \texttt{mtcars} dataset, which
was extracted from the 1974 \emph{Motor Trend} US magazine. The dataset
contains fuel consumption and 10 characteristics of automobile design
and performance for 32 different car models (see \texttt{?mtcars} for
details). We chose this dataset because it is accessible from base R and
familiar to many R users. We might want to conduct specific statistical
analyses on this data set, say, \emph{t} tests or structural equation
modelling, but first, we want to check for outliers that may influence
those test results.

Because the automobile names are stored as column names in
\texttt{mtcars}, we first have to convert them to an ID column to
benefit from the \texttt{check\_outliers()} ID argument. Furthermore, we
only really need a couple columns for this demonstration, so we choose
the first four (\texttt{mpg} = Miles/(US) gallon; \texttt{cyl} = Number
of cylinders; \texttt{disp} = Displacement; \texttt{hp} = Gross
horsepower). Finally, because there are no outliers in this dataset, we
add two artificial outliers before running our function.

\begin{Shaded}
\begin{Highlighting}[]
\FunctionTok{library}\NormalTok{(performance)}

\CommentTok{\# Create some artificial outliers and an ID column}
\NormalTok{data }\OtherTok{\textless{}{-}} \FunctionTok{rbind}\NormalTok{(mtcars[}\DecValTok{1}\SpecialCharTok{:}\DecValTok{4}\NormalTok{], }\DecValTok{42}\NormalTok{, }\DecValTok{55}\NormalTok{)}
\NormalTok{data }\OtherTok{\textless{}{-}} \FunctionTok{cbind}\NormalTok{(}\AttributeTok{car =} \FunctionTok{row.names}\NormalTok{(data), data)}

\NormalTok{outliers }\OtherTok{\textless{}{-}} \FunctionTok{check\_outliers}\NormalTok{(data, }\AttributeTok{method =} \StringTok{"zscore\_robust"}\NormalTok{, }\AttributeTok{ID =} \StringTok{"car"}\NormalTok{)}
\NormalTok{outliers}
\end{Highlighting}
\end{Shaded}

\begin{verbatim}
#> 2 outliers detected: cases 33, 34.
#> - Based on the following method and threshold: zscore_robust (3.291).
#> - For variables: mpg, cyl, disp, hp.
#> 
#> -----------------------------------------------------------------------------
#>  
#> The following observations were considered outliers for two or more
#>   variables by at least one of the selected methods:
#> 
#>   Row car n_Zscore_robust
#> 1  33  33               2
#> 2  34  34               2
#> 
#> -----------------------------------------------------------------------------
#> Outliers per variable (zscore_robust): 
#> 
#> $mpg
#>    Row car Distance_Zscore_robust
#> 33  33  33               3.709699
#> 34  34  34               5.848328
#> 
#> $cyl
#>    Row car Distance_Zscore_robust
#> 33  33  33               12.14083
#> 34  34  34               16.52502
\end{verbatim}

What we see is that \texttt{check\_outliers()} with the robust \emph{z}
score method detected two outliers: cases 33 and 34, which were the
observations we added ourselves. They were flagged for two variables
specifically: \texttt{mpg} (Miles/(US) gallon) and \texttt{cyl} (Number
of cylinders), and the output provides their exact \emph{z} score for
those variables.

We describe how to deal with those cases in more details later in the
paper, but should we want to exclude these detected outliers from the
main dataset, we can extract row numbers using \texttt{which()} on the
output object, which can then be used for indexing:

\begin{Shaded}
\begin{Highlighting}[]
\FunctionTok{which}\NormalTok{(outliers)}
\end{Highlighting}
\end{Shaded}

\begin{verbatim}
#> [1] 33 34
\end{verbatim}

\begin{Shaded}
\begin{Highlighting}[]
\NormalTok{data\_clean }\OtherTok{\textless{}{-}}\NormalTok{ data[}\SpecialCharTok{{-}}\FunctionTok{which}\NormalTok{(outliers), ]}
\end{Highlighting}
\end{Shaded}

All \texttt{check\_outliers()} output objects possess a \texttt{plot()}
method, meaning it is also possible to visualize the outliers using the
generic \texttt{plot()} function on the resulting outlier object after
loading the \{see\} package (Figure 1).

\begin{Shaded}
\begin{Highlighting}[]
\FunctionTok{library}\NormalTok{(see)}

\FunctionTok{plot}\NormalTok{(outliers)}
\end{Highlighting}
\end{Shaded}

\begin{figure}
\includegraphics[width=1\linewidth]{paper_files/figure-latex/univariate_implicit-1} \caption{Visual depiction of outliers using the robust z-score method. The distance represents an aggregate score for variables mpg, cyl, disp, and hp.}\label{fig:univariate_implicit}
\end{figure}

Other univariate methods are available, such as using the interquartile
range (IQR), or based on different intervals, such as the Highest
Density Interval (HDI) or the Bias Corrected and Accelerated Interval
(BCI). These methods are documented and described in the function's
\href{https://easystats.github.io/performance/reference/check_outliers.html}{help
page}.

\hypertarget{multivariate-outliers}{%
\subsection{Multivariate Outliers}\label{multivariate-outliers}}

Univariate outliers can be useful when the focus is on a particular
variable, for instance the reaction time, as extreme values might be
indicative of inattention or non-task-related behavior\footnote{ Note
  that they might not be the optimal way of treating reaction time
  outliers \citep{ratcliff1993methods, van1995statistical}}.

However, in many scenarios, variables of a data set are not independent,
and an abnormal observation will impact multiple dimensions. For
instance, a participant giving random answers to a questionnaire. In
this case, computing the \emph{z} score for each of the questions might
not lead to satisfactory results. Instead, one might want to look at
these variables together.

One common approach for this is to compute multivariate distance metrics
such as the Mahalanobis distance. Although the Mahalanobis distance is
very popular, just like the regular \emph{z} scores method, it is not
robust and is heavily influenced by the outliers themselves. Therefore,
for multivariate outliers, it is recommended to use the Minimum
Covariance Determinant, a robust version of the Mahalanobis distance
\citep[MCD,][]{leys2018outliers, leys2019outliers}.

In \emph{\{performance\}}'s \texttt{check\_outliers()}, one can use this
approach with \texttt{method\ =\ "mcd"}.\footnote{Our default threshold
  for the MCD method is defined by
  \texttt{stats::qchisq(p\ =\ 1\ -\ 0.001,\ df\ =\ ncol(x))}, which
  again is an approximation of the critical value for \emph{p}
  \textless{} .001 consistent with the thresholds of our other methods.}

\begin{Shaded}
\begin{Highlighting}[]
\NormalTok{outliers }\OtherTok{\textless{}{-}} \FunctionTok{check\_outliers}\NormalTok{(data, }\AttributeTok{method =} \StringTok{"mcd"}\NormalTok{)}
\NormalTok{outliers}
\end{Highlighting}
\end{Shaded}

\begin{verbatim}
#> 9 outliers detected: cases 7, 15, 16, 17, 24, 29, 31, 33, 34.
#> - Based on the following method and threshold: mcd (20).
#> - For variables: mpg, cyl, disp, hp.
\end{verbatim}

Here, we detected 9 multivariate outliers (i.e,. when looking at all
variables of our dataset together).

\begin{Shaded}
\begin{Highlighting}[]
\FunctionTok{plot}\NormalTok{(outliers)}
\end{Highlighting}
\end{Shaded}

\begin{figure}
\includegraphics[width=1\linewidth]{paper_files/figure-latex/multivariate_implicit-1} \caption{Visual depiction of outliers using the Minimum Covariance Determinant (MCD) method, a robust version of the Mahalanobis distance. The distance represents the MCD scores for variables mpg, cyl, disp, and hp.}\label{fig:multivariate_implicit}
\end{figure}

Other multivariate methods are available, such as another type of robust
Mahalanobis distance that in this case relies on an orthogonalized
Gnanadesikan-Kettenring pairwise estimator
\citep{gnanadesikan1972robust}. These methods are documented and
described in the function's
\href{https://easystats.github.io/performance/reference/check_outliers.html}{help
page}.

\hypertarget{model-based-outliers}{%
\subsection{Model-Based Outliers}\label{model-based-outliers}}

Working with regression models creates the possibility of using
model-based SOD methods. These methods rely on the concept of
\emph{leverage}, that is, how much influence a given observation can
have on the model estimates. If few observations have a relatively
strong leverage/influence on the model, one can suspect that the model's
estimates are biased by these observations, in which case flagging them
as outliers could prove helpful (see next section, ``Handling
Outliers'').

In \{performance\}, two such model-based SOD methods are currently
available: Cook's distance, for regular regression models, and Pareto,
for Bayesian models. As such, \texttt{check\_outliers()} can be applied
directly on regression model objects, by simply specifying
\texttt{method\ =\ "cook"} (or \texttt{method\ =\ "pareto"} for Bayesian
models).\footnote{Our default threshold for the Cook method is defined
  by \texttt{stats::qf(0.5,\ ncol(x),\ nrow(x)\ -\ ncol(x))}, which
  again is an approximation of the critical value for \emph{p}
  \textless{} .001 consistent with the thresholds of our other methods.}

Currently, most lm models are supported (with the exception of
\texttt{glmmTMB}, \texttt{lmrob}, and \texttt{glmrob} models), as long
as they are supported by the underlying functions
\texttt{stats::cooks.distance()} (or \texttt{loo::pareto\_k\_values()})
and \texttt{insight::get\_data()} (for a full list of the 225 models
currently supported by the \texttt{insight} package, see
\url{https://easystats.github.io/insight/\#list-of-supported-models-by-class}).
Also note that although \texttt{check\_outliers()} supports the pipe
operators (\texttt{\textbar{}\textgreater{}} or
\texttt{\%\textgreater{}\%}), it does not support \texttt{tidymodels} at
this time. We show a demo below.

\begin{Shaded}
\begin{Highlighting}[]
\NormalTok{model }\OtherTok{\textless{}{-}} \FunctionTok{lm}\NormalTok{(disp }\SpecialCharTok{\textasciitilde{}}\NormalTok{ mpg }\SpecialCharTok{*}\NormalTok{ disp, }\AttributeTok{data =}\NormalTok{ data)}
\NormalTok{outliers }\OtherTok{\textless{}{-}} \FunctionTok{check\_outliers}\NormalTok{(model, }\AttributeTok{method =} \StringTok{"cook"}\NormalTok{)}
\NormalTok{outliers}
\end{Highlighting}
\end{Shaded}

\begin{verbatim}
#> 1 outlier detected: case 34.
#> - Based on the following method and threshold: cook (0.708).
#> - For variable: (Whole model).
\end{verbatim}

Using the model-based outlier detection method, we identified a single
outlier.

\begin{Shaded}
\begin{Highlighting}[]
\NormalTok{model }\OtherTok{\textless{}{-}} \FunctionTok{lm}\NormalTok{(disp }\SpecialCharTok{\textasciitilde{}}\NormalTok{ mpg }\SpecialCharTok{*}\NormalTok{ disp, }\AttributeTok{data =}\NormalTok{ data)}
\NormalTok{outliers }\OtherTok{\textless{}{-}} \FunctionTok{check\_outliers}\NormalTok{(model, }\AttributeTok{method =} \StringTok{"cook"}\NormalTok{)}
\NormalTok{outliers}
\end{Highlighting}
\end{Shaded}

\begin{verbatim}
#> 1 outlier detected: case 34.
#> - Based on the following method and threshold: cook (0.708).
#> - For variable: (Whole model).
\end{verbatim}

\begin{Shaded}
\begin{Highlighting}[]
\FunctionTok{plot}\NormalTok{(outliers)}
\end{Highlighting}
\end{Shaded}

\begin{figure}
\includegraphics[width=1\linewidth]{paper_files/figure-latex/model_fig-1} \caption{Visual depiction of outliers based on Cook's distance (leverage and standardized residuals), based on the fitted model.}\label{fig:model_fig}
\end{figure}

Table 1 below summarizes which methods to use in which cases, and with
what threshold. The recommended thresholds are the default thresholds.

\begin{longtable}[]{@{}
  >{\raggedright\arraybackslash}p{(\columnwidth - 6\tabcolsep) * \real{0.2735}}
  >{\raggedright\arraybackslash}p{(\columnwidth - 6\tabcolsep) * \real{0.2466}}
  >{\raggedright\arraybackslash}p{(\columnwidth - 6\tabcolsep) * \real{0.2601}}
  >{\raggedright\arraybackslash}p{(\columnwidth - 6\tabcolsep) * \real{0.2197}}@{}}
\caption{Summary of Statistical Outlier Detection Methods
Recommendations}\tabularnewline
\toprule\noalign{}
\begin{minipage}[b]{\linewidth}\raggedright
Statistical Test
\end{minipage} & \begin{minipage}[b]{\linewidth}\raggedright
Diagnosis Method
\end{minipage} & \begin{minipage}[b]{\linewidth}\raggedright
Recommended Threshold
\end{minipage} & \begin{minipage}[b]{\linewidth}\raggedright
Function Usage
\end{minipage} \\
\midrule\noalign{}
\endfirsthead
\toprule\noalign{}
\begin{minipage}[b]{\linewidth}\raggedright
Statistical Test
\end{minipage} & \begin{minipage}[b]{\linewidth}\raggedright
Diagnosis Method
\end{minipage} & \begin{minipage}[b]{\linewidth}\raggedright
Recommended Threshold
\end{minipage} & \begin{minipage}[b]{\linewidth}\raggedright
Function Usage
\end{minipage} \\
\midrule\noalign{}
\endhead
\bottomrule\noalign{}
\endlastfoot
Supported regression model & \textbf{Model-based}: Cook (or Pareto for
Bayesian models) & \emph{qf(0.5, ncol(x), nrow(x) - ncol(x))} (or 0.7
for Pareto) & \emph{check\_outliers(model, method = ``cook'')} \\
& & & \\
Structural Equation Modeling (or other unsupported model) &
\textbf{Multivariate}: Minimum Covariance Determinant (MCD) &
\emph{qchisq(p = 1 - 0.001, df = ncol(x))} & \emph{check\_outliers(data,
method = ``mcd'')} \\
& & & \\
Simple test with few variables (\emph{t} test, correlation, etc.) &
\textbf{Univariate}: robust \emph{z} scores (MAD) & \emph{qnorm(p = 1 -
0.001 / 2)}, \textasciitilde{} 3.29 & \emph{check\_outliers(data, method
= ``zscore\_robust'')} \\
\end{longtable}

\hypertarget{cooks-distance-vs.-mcd}{%
\subsection{Cook's Distance vs.~MCD}\label{cooks-distance-vs.-mcd}}

\citet{leys2018outliers} report a preference for the MCD method over
Cook's distance. This is because Cook's distance removes one observation
at a time and checks its corresponding influence on the model each time
\citep{cook1977detection}, and flags any observation that has a large
influence. In the view of these authors, when there are several
outliers, the process of removing a single outlier at a time is
problematic as the model remains ``contaminated'' or influenced by other
possible outliers in the model, rendering this method suboptimal in the
presence of multiple outliers.

However, distribution-based approaches are not a silver bullet either,
and there are cases where the usage of methods agnostic to theoretical
and statistical models of interest might be problematic. For example, a
very tall person would be expected to also be much heavier than average,
but that would still fit with the expected association between height
and weight (i.e., it would be in line with a model such as
\texttt{weight\ \textasciitilde{}\ height}). In contrast, using
multivariate outlier detection methods there may flag this person as
being an outlier---being unusual on two variables, height and
weight---even though the pattern fits perfectly with our predictions.

In the example below, we plot the raw data and see two possible
outliers. The first one falls along the regression line, and is
therefore ``in line'' with our hypothesis. The second one clearly
diverges from the regression line, and therefore we can conclude that
this outlier may have a disproportionate influence on our model.

\begin{Shaded}
\begin{Highlighting}[]
\NormalTok{data }\OtherTok{\textless{}{-}}\NormalTok{ women[}\FunctionTok{rep}\NormalTok{(}\FunctionTok{seq\_len}\NormalTok{(}\FunctionTok{nrow}\NormalTok{(women)), }\AttributeTok{each =} \DecValTok{100}\NormalTok{), ]}
\NormalTok{data }\OtherTok{\textless{}{-}} \FunctionTok{rbind}\NormalTok{(data, }\FunctionTok{c}\NormalTok{(}\DecValTok{100}\NormalTok{, }\DecValTok{258}\NormalTok{), }\FunctionTok{c}\NormalTok{(}\DecValTok{100}\NormalTok{, }\DecValTok{200}\NormalTok{))}
\NormalTok{model }\OtherTok{\textless{}{-}} \FunctionTok{lm}\NormalTok{(weight }\SpecialCharTok{\textasciitilde{}}\NormalTok{ height, data)}
\NormalTok{rempsyc}\SpecialCharTok{::}\FunctionTok{nice\_scatter}\NormalTok{(data, }\StringTok{"height"}\NormalTok{, }\StringTok{"weight"}\NormalTok{)}
\end{Highlighting}
\end{Shaded}

\begin{figure}
\includegraphics[width=1\linewidth]{paper_files/figure-latex/scatter-1} \caption{Scatter plot of height and weight, with two extreme observations: one model-consistent (top-right) and the other, model-inconsistent (i.e., an outlier; bottom-right).}\label{fig:scatter}
\end{figure}

Using either the \emph{z}-score or MCD methods, our model-consistent
observation will be incorrectly flagged as an outlier or influential
observation.

\begin{Shaded}
\begin{Highlighting}[]
\NormalTok{outliers }\OtherTok{\textless{}{-}} \FunctionTok{check\_outliers}\NormalTok{(model, }\AttributeTok{method =} \FunctionTok{c}\NormalTok{(}\StringTok{"zscore\_robust"}\NormalTok{, }\StringTok{"mcd"}\NormalTok{))}
\FunctionTok{which}\NormalTok{(outliers)}
\end{Highlighting}
\end{Shaded}

\begin{verbatim}
#> [1] 1501 1502
\end{verbatim}

In contrast, the model-based detection method displays the desired
behaviour: it correctly flags the person who is very tall but very
light, without flagging the person who is both tall and heavy.

\begin{Shaded}
\begin{Highlighting}[]
\NormalTok{outliers }\OtherTok{\textless{}{-}} \FunctionTok{check\_outliers}\NormalTok{(model, }\AttributeTok{method =} \StringTok{"cook"}\NormalTok{)}
\FunctionTok{which}\NormalTok{(outliers)}
\end{Highlighting}
\end{Shaded}

\begin{verbatim}
#> [1] 1502
\end{verbatim}

\begin{Shaded}
\begin{Highlighting}[]
\FunctionTok{plot}\NormalTok{(outliers)}
\end{Highlighting}
\end{Shaded}

\begin{figure}
\includegraphics[width=1\linewidth]{paper_files/figure-latex/model3-1} \caption{The leverage method (Cook's distance) correctly distinguishes the true outlier from the model-consistent extreme observation), based on the fitted model.}\label{fig:model3}
\end{figure}

Finally, unusual observations happen naturally: extreme observations are
expected even when taken from a normal distribution. While statistical
models can integrate this ``expectation'', multivariate outlier methods
might be too conservative, flagging too many observations despite
belonging to the right generative process. For these reasons, we believe
that model-based methods are still preferable to the MCD when using
supported regression models. Additionally, if the presence of multiple
outliers is a significant concern, regression methods that are more
robust to outliers should be considered---like \emph{t} regression or
quantile regression---as they render their precise identification less
critical \citep{mcelreath2020statistical}.

\hypertarget{composite-outlier-score}{%
\subsection{Composite Outlier Score}\label{composite-outlier-score}}

The \emph{\{performance\}} package also offers an alternative,
consensus-based approach that combines several methods, based on the
assumption that different methods provide different angles of looking at
a given problem. By applying a variety of methods, one can hope to
``triangulate'' the true outliers (those consistently flagged by
multiple methods) and thus attempt to minimize false positives.

In practice, this approach computes a composite outlier score, formed of
the average of the binary (0 or 1) classification results of each
method. It represents the probability that each observation is
classified as an outlier by at least one method. The default decision
rule classifies rows with composite outlier scores superior or equal to
0.5 as outlier observations (i.e., that were classified as outliers by
at least half of the methods). In \emph{\{performance\}}'s
\texttt{check\_outliers()}, one can use this approach by including all
desired methods in the corresponding argument.

\begin{Shaded}
\begin{Highlighting}[]
\NormalTok{outliers }\OtherTok{\textless{}{-}} \FunctionTok{check\_outliers}\NormalTok{(model, }\AttributeTok{method =} \FunctionTok{c}\NormalTok{(}\StringTok{"zscore\_robust"}\NormalTok{, }\StringTok{"mcd"}\NormalTok{, }\StringTok{"cook"}\NormalTok{))}
\FunctionTok{which}\NormalTok{(outliers)}
\end{Highlighting}
\end{Shaded}

\begin{verbatim}
#> [1] 1501 1502
\end{verbatim}

Outliers (counts or per variables) for individual methods can then be
obtained through attributes. For example:

\begin{Shaded}
\begin{Highlighting}[]
\FunctionTok{attributes}\NormalTok{(outliers)}\SpecialCharTok{$}\NormalTok{outlier\_var}\SpecialCharTok{$}\NormalTok{zscore\_robust}
\end{Highlighting}
\end{Shaded}

\begin{verbatim}
#> $weight
#>       Row Distance_Zscore_robust
#> 1501 1501               6.913530
#> 1502 1502               3.653492
#> 
#> $height
#>       Row Distance_Zscore_robust
#> 1501 1501               5.901794
#> 1502 1502               5.901794
\end{verbatim}

An example sentence for reporting the usage of the composite method
could be:

\begin{quote}
Based on a composite outlier score \citep[see the `check\_outliers()'
function in the `performance' R package,][]{ludecke2021performance}
obtained via the joint application of multiple outliers detection
algorithms \citetext{\citealp[(a) median absolute deviation (MAD)-based
robust \emph{z} scores,][]{leys2013outliers}; \citealp[(b) Mahalanobis
minimum covariance determinant (MCD),][]{leys2019outliers}; \citealp[and
(c) Cook's distance,][]{cook1977detection}}, we excluded two
participants that were classified as outliers by at least half of the
methods used.
\end{quote}

\hypertarget{handling-outliers}{%
\section{Handling Outliers}\label{handling-outliers}}

The above section demonstrated how to identify outliers using the
\texttt{check\_outliers()} function in the \emph{\{performance\}}
package. But what should we do with these outliers once identified?
Although it is common to automatically discard any observation that has
been marked as ``an outlier'' as if it might infect the rest of the data
with its statistical ailment, we believe that the use of SOD methods is
but one step in the get-to-know-your-data pipeline; a researcher or
analyst's \emph{domain knowledge} must be involved in the decision of
how to deal with observations marked as outliers by means of SOD.
Indeed, automatic tools can help detect outliers, but they are nowhere
near perfect. Although they can be useful to flag suspect data, they can
have misses and false alarms, and they cannot replace human eyes and
proper vigilance from the researcher.

For example, in the case of reaction time analysis,
\citet{miller2023outlier} systematically compared 58 SOD procedures in
simulations using large datasets of real reaction times. He concluded
that regardless of the selected procedure, the exclusion of outliers
(reaction times too slow or too fast) generally did more harm than good
compared to retaining them. He thus recommends only excluding reaction
times that are clearly invalid, such as those under a fixed threshold,
e.g., 150 ms, which is close to the minimal physiological limit for
reacting to a visual stimulus. Setting an upper limit on very long times
(e.g., 3 to 5 seconds, depending on the experimental task) to remove
potential sparse artifacts, can also improve model convergence and
fitting.

\citet{miller2023outlier} also suggests that it is generally better to
assess outliers within specific experimental conditions or groups (a
condition-specific strategy), rather than across the entire dataset at
once (a pooled strategy), particularly in the case of reaction times.
Additionally, common procedures such as statistical transformations
(e.g.~log-transformation) reportedly offer at best no benefit (being
instead potentially detrimental) to statistical power
\citep{schramm2019reaction}. Given the specific shape of a typical
reaction distribution, treating them with bespoke models that take into
account its skewness (thus reframing the notion of outliers and
integrating the longer right tail of the distribution) should be
considered. Examples of such models---referred to as sequential sampling
models or evidence accumulation models---include Wald models
\citep{anders2016shifted}, log-normal race models
\citep{rouder2015lognormal}, Linear Ballistic Accumulators
\citep{brown2008simplest}, and Drift Diffusion Models
\citep{ratcliff2016diffusion}.

Thus, when manually inspecting data for outliers, it can be helpful to
think of outliers as belonging to different types of outliers, or
categories, which can help decide what to do with a given outlier.

\hypertarget{error-interesting-and-random-outliers}{%
\subsection{Error, Interesting, and Random
Outliers}\label{error-interesting-and-random-outliers}}

\citet{leys2019outliers} distinguish between error outliers, interesting
outliers, and random outliers. \emph{Error outliers} are likely due to
human error and should be corrected before data analysis or outright
removed since they are invalid observations (e.g., physiologically
implausible reaction times). \emph{Interesting outliers} are not due to
technical error and may be of theoretical interest; it might thus be
relevant to investigate them further even though they should be removed
from the current analysis of interest. \emph{Random outliers} are
assumed to be due to chance alone and to belong to the correct
distribution and, therefore, should be retained.

It is recommended to \emph{keep} observations which are expected to be
part of the distribution of interest, even if they are outliers
\citep{leys2019outliers}. However, if it is suspected that the outliers
belong to an alternative distribution, then those observations could
have a large impact on the results and call into question their
robustness, especially if significance is conditional on their
inclusion, so should be removed.

We should also keep in mind that there might be error outliers that are
not detected by statistical tools, but should nonetheless be found and
removed. For example, if we are studying the effects of X on Y among
teenagers and we have one observation from a 20-year-old, this
observation might not be a \emph{statistical outlier}, but it is an
outlier in the \emph{context} of our research, and should be discarded.
We could call these observations \emph{undetected} error outliers, in
the sense that although they do not statistically stand out, they do not
belong to the theoretical or empirical distribution of interest (e.g.,
teenagers). In this way, we should not blindly rely on statistical
outlier detection methods; doing our due diligence to investigate
undetected error outliers relative to our specific research question is
also essential for valid inferences.

\hypertarget{winsorization}{%
\subsection{Winsorization}\label{winsorization}}

\emph{Removing} outliers that do not belong to the distribution of
interest can in this case be a valid strategy, and ideally one would
report results with and without outliers to see the extent of their
impact on results. This approach however can reduce statistical power.
Therefore, some propose a \emph{recoding} approach, namely,
winsorization: bringing outliers back within acceptable limits
\citep[e.g., 3 MADs,][]{tukey1963less}. However, if possible, it is
recommended to collect enough data so that even after removing outliers,
there is still sufficient statistical power without having to resort to
winsorization \citep{leys2019outliers}.

The \emph{easystats} ecosystem makes it easy to incorporate this step
into your workflow through the \texttt{winsorize()} function of
\emph{\{datawizard\}}, a lightweight R package to facilitate data
wrangling and statistical transformations \citep{patil2022datawizard}.
This procedure will bring back univariate outliers within the limits of
`acceptable' values, based either on the percentile, the \emph{z} score,
or its robust alternative based on the MAD. For example, let's say we
want to winsorize the two outliers identified before:

\begin{Shaded}
\begin{Highlighting}[]
\NormalTok{data[}\DecValTok{1501}\SpecialCharTok{:}\DecValTok{1502}\NormalTok{, ]  }\CommentTok{\# See outliers rows}
\end{Highlighting}
\end{Shaded}

\begin{verbatim}
#>      height weight
#> 1501    100    258
#> 1502    100    200
\end{verbatim}

\begin{Shaded}
\begin{Highlighting}[]
\CommentTok{\# Winsorizing using the MAD}
\FunctionTok{library}\NormalTok{(datawizard)}
\NormalTok{winsorized\_data }\OtherTok{\textless{}{-}} \FunctionTok{winsorize}\NormalTok{(data, }\AttributeTok{method =} \StringTok{"zscore"}\NormalTok{, }\AttributeTok{robust =} \ConstantTok{TRUE}\NormalTok{, }\AttributeTok{threshold =} \DecValTok{3}\NormalTok{)}

\CommentTok{\# Values \textgreater{} +/{-} MAD have been winsorized}
\NormalTok{winsorized\_data[}\DecValTok{1501}\SpecialCharTok{:}\DecValTok{1502}\NormalTok{, ]}
\end{Highlighting}
\end{Shaded}

\begin{verbatim}
#>       height   weight
#> 1501 82.7912 188.3736
#> 1502 82.7912 188.3736
\end{verbatim}

\hypertarget{the-importance-of-transparency}{%
\subsection{The Importance of
Transparency}\label{the-importance-of-transparency}}

Finally, it is a critical part of a sound outlier treatment that
regardless of which SOD method used, it should be reported in a
reproducible manner. Ideally, the handling of outliers should be
specified \emph{a priori} with as much detail as possible, and
preregistered, to limit researchers' degrees of freedom and therefore
risks of false positives \citep{leys2019outliers}. This is especially
true given that interesting outliers and random outliers are often times
hard to distinguish in practice. Thus, researchers should always
prioritize transparency and report all of the following information: (a)
how many outliers were identified (including percentage); (b) according
to which method and criteria, (c) using which function of which R
package (if applicable), and (d) how they were handled (excluded or
winsorized, if the latter, using what threshold). If at all possible,
(e) the corresponding code script along with the data should be shared
on a public repository like the Open Science Framework (OSF), so that
the exclusion criteria can be reproduced precisely.

\hypertarget{conclusion}{%
\section{Conclusion}\label{conclusion}}

In this paper, we have showed how to investigate outliers using the
\texttt{check\_outliers()} function of the \emph{\{performance\}}
package while following current good practices. However, best practice
for outlier treatment does not stop at using appropriate statistical
algorithms, but entails respecting existing recommendations, such as
preregistration, reproducibility, consistency, transparency, and
justification. Ideally, one would additionally also report the package,
function, and threshold used (linking to the full code when possible).
We hope that this paper and the accompanying \texttt{check\_outlier()}
function of \emph{easystats} will help researchers engage in good
research practices while providing a smooth outlier detection
experience.

\hypertarget{contributions}{%
\subsubsection{Contributions}\label{contributions}}

RT: Writing- Original draft preparation, Writing- Reviewing and Editing,
Software. MSB-S, IP, DL, BMW, and DM: Writing- Reviewing and Editing,
Software.

\hypertarget{acknowledgements}{%
\subsubsection{Acknowledgements}\label{acknowledgements}}

\emph{\{performance\}} is part of the collaborative
\href{https://github.com/easystats/easystats}{\emph{easystats}}
ecosystem \citep{easystatspackage}. Thus, we thank all
\href{https://github.com/orgs/easystats/people}{members of easystats},
contributors, and users alike.

\hypertarget{funding-information}{%
\subsubsection{Funding information}\label{funding-information}}

This research received no external funding.

\hypertarget{competing-interests}{%
\subsubsection{Competing Interests}\label{competing-interests}}

The authors declare no conflict of interest

\renewcommand\refname{References}
\bibliography{paper.bib}


\end{document}
